% Options for packages loaded elsewhere
\PassOptionsToPackage{unicode}{hyperref}
\PassOptionsToPackage{hyphens}{url}
\documentclass[
  10pt,
]{article}
\usepackage{xcolor}
\usepackage[margin=0.8in]{geometry}
\usepackage{amsmath,amssymb}
\setcounter{secnumdepth}{-\maxdimen} % remove section numbering
\usepackage{iftex}
\ifPDFTeX
  \usepackage[T1]{fontenc}
  \usepackage[utf8]{inputenc}
  \usepackage{textcomp} % provide euro and other symbols
\else % if luatex or xetex
  \usepackage{unicode-math} % this also loads fontspec
  \defaultfontfeatures{Scale=MatchLowercase}
  \defaultfontfeatures[\rmfamily]{Ligatures=TeX,Scale=1}
\fi
\usepackage{lmodern}
\ifPDFTeX\else
  % xetex/luatex font selection
\fi
% Use upquote if available, for straight quotes in verbatim environments
\IfFileExists{upquote.sty}{\usepackage{upquote}}{}
\IfFileExists{microtype.sty}{% use microtype if available
  \usepackage[]{microtype}
  \UseMicrotypeSet[protrusion]{basicmath} % disable protrusion for tt fonts
}{}
\makeatletter
\@ifundefined{KOMAClassName}{% if non-KOMA class
  \IfFileExists{parskip.sty}{%
    \usepackage{parskip}
  }{% else
    \setlength{\parindent}{0pt}
    \setlength{\parskip}{6pt plus 2pt minus 1pt}}
}{% if KOMA class
  \KOMAoptions{parskip=half}}
\makeatother
\usepackage{graphicx}
\makeatletter
\newsavebox\pandoc@box
\newcommand*\pandocbounded[1]{% scales image to fit in text height/width
  \sbox\pandoc@box{#1}%
  \Gscale@div\@tempa{\textheight}{\dimexpr\ht\pandoc@box+\dp\pandoc@box\relax}%
  \Gscale@div\@tempb{\linewidth}{\wd\pandoc@box}%
  \ifdim\@tempb\p@<\@tempa\p@\let\@tempa\@tempb\fi% select the smaller of both
  \ifdim\@tempa\p@<\p@\scalebox{\@tempa}{\usebox\pandoc@box}%
  \else\usebox{\pandoc@box}%
  \fi%
}
% Set default figure placement to htbp
\def\fps@figure{htbp}
\makeatother
\setlength{\emergencystretch}{3em} % prevent overfull lines
\providecommand{\tightlist}{%
  \setlength{\itemsep}{0pt}\setlength{\parskip}{0pt}}
\usepackage{bookmark}
\IfFileExists{xurl.sty}{\usepackage{xurl}}{} % add URL line breaks if available
\urlstyle{same}
\hypersetup{
  pdftitle={Assignment 4},
  hidelinks,
  pdfcreator={LaTeX via pandoc}}

\title{Assignment 4}
\author{}
\date{\vspace{-2.5em}}

\begin{document}
\maketitle

\begin{itemize}
\item
  For this assignment, I will be analyzing data from the barroso2021
  data set. This data set is from the psymetadata package, and it
  observes the relationship between math achievement and math anxiety
  (\url{https://psycnet.apa.org/record/2020-80018-001}). Participants
  are in grades 1 through 6, and math ability is characterized as either
  low or not low. Grade 5 has a significantly larger sample size than
  the other grades, which will likely impact effect size measurements.
  Note: only grades 2 and 5 have students in the low math ability
  condition.
\item
  The focus of this analysis will be analyzing differences in effect
  size across different parameters, where the absolute value was
  calculated, so values range from 0 (no effect) to 1 (strongest
  possible effect)
\end{itemize}

\pandocbounded{\includegraphics[keepaspectratio]{Assignment4_Stein_files/figure-latex/unnamed-chunk-1-1.pdf}}

\pandocbounded{\includegraphics[keepaspectratio]{Assignment4_Stein_files/figure-latex/unnamed-chunk-2-1.pdf}}

\section{Conclusion}\label{conclusion}

There is a significantly smaller effect size for low math ability
participants compared to average-above average students. This is likely
impacted by the small sample sizes in these conditions. There is a less
obvious (but still important) difference in effect sizes for each grade.
Grade 5 has the highest number of scores for each effect size condition
except ``large effect''. This is likely due to its significantly higher
sample size than that of the other grades.

\end{document}
