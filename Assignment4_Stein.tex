\documentclass[10pt,]{article}
\usepackage[margin=0.8in]{geometry}
\newcommand*{\authorfont}{\fontfamily{phv}\selectfont}
\usepackage{lmodern}
\usepackage{abstract}
\renewcommand{\abstractname}{}    % clear the title
\renewcommand{\absnamepos}{empty} % originally center
\newcommand{\blankline}{\quad\pagebreak[2]}

\providecommand{\tightlist}{%
  \setlength{\itemsep}{0pt}\setlength{\parskip}{0pt}} 
\usepackage{longtable,booktabs}

\usepackage{parskip}
\usepackage{titlesec}
\titlespacing\section{0pt}{12pt plus 4pt minus 2pt}{6pt plus 2pt minus 2pt}
\titlespacing\subsection{0pt}{12pt plus 4pt minus 2pt}{6pt plus 2pt minus 2pt}

\usepackage{titling}
\setlength{\droptitle}{-.25cm}

%\setlength{\parindent}{0pt}
%\setlength{\parskip}{6pt plus 2pt minus 1pt}
%\setlength{\emergencystretch}{3em}  % prevent overfull lines 
\setlength{\parskip}{12pt} 

\usepackage[T1]{fontenc}
\usepackage[utf8]{inputenc}
\linespread{1.05}

\usepackage{fancyhdr}
\pagestyle{fancy}
\usepackage{lastpage}
\renewcommand{\headrulewidth}{0.0pt}
\renewcommand{\footrulewidth}{0.0pt} 

\lhead{}
\chead{}

\rhead{}


\fancypagestyle{firststyle}
{
\renewcommand{\headrulewidth}{0pt}%
   \fancyhf{}
   \fancyfoot[C]{\small \thepage/\pageref*{LastPage}}
}

%\def\labelitemi{--}
%\usepackage{enumitem}
%\setitemize[0]{leftmargin=25pt}
%\setenumerate[0]{leftmargin=25pt}


\usepackage{titlesec}

\titleformat*{\subsection}{\itshape}
\titleformat*{\subsubsection}{\itshape}

\makeatletter
\@ifpackageloaded{hyperref}{}{%
\ifxetex
  \usepackage[setpagesize=false, % page size defined by xetex
              unicode=false, % unicode breaks when used with xetex
              xetex]{hyperref}
\else
  \usepackage[unicode=true]{hyperref}
\fi
}
\@ifpackageloaded{color}{
    \PassOptionsToPackage{usenames,dvipsnames}{color}
}{%
    \usepackage[usenames,dvipsnames]{color}
}
\makeatother
\hypersetup{breaklinks=true,
            bookmarks=true,
            pdfauthor={},
             pdfkeywords = {},  
            pdftitle={},
            colorlinks=true,
            citecolor=blue,
            urlcolor=blue,
            linkcolor=magenta,
            pdfborder={0 0 0}}
\urlstyle{same}  % don't use monospace font for urls


\setcounter{secnumdepth}{0}

\usepackage{color}
\usepackage{fancyvrb}
\newcommand{\VerbBar}{|}
\newcommand{\VERB}{\Verb[commandchars=\\\{\}]}
\DefineVerbatimEnvironment{Highlighting}{Verbatim}{commandchars=\\\{\}}
% Add ',fontsize=\small' for more characters per line
\usepackage{framed}
\definecolor{shadecolor}{RGB}{248,248,248}
\newenvironment{Shaded}{\begin{snugshade}}{\end{snugshade}}
\newcommand{\AlertTok}[1]{\textcolor[rgb]{0.94,0.16,0.16}{#1}}
\newcommand{\AnnotationTok}[1]{\textcolor[rgb]{0.56,0.35,0.01}{\textbf{\textit{#1}}}}
\newcommand{\AttributeTok}[1]{\textcolor[rgb]{0.13,0.29,0.53}{#1}}
\newcommand{\BaseNTok}[1]{\textcolor[rgb]{0.00,0.00,0.81}{#1}}
\newcommand{\BuiltInTok}[1]{#1}
\newcommand{\CharTok}[1]{\textcolor[rgb]{0.31,0.60,0.02}{#1}}
\newcommand{\CommentTok}[1]{\textcolor[rgb]{0.56,0.35,0.01}{\textit{#1}}}
\newcommand{\CommentVarTok}[1]{\textcolor[rgb]{0.56,0.35,0.01}{\textbf{\textit{#1}}}}
\newcommand{\ConstantTok}[1]{\textcolor[rgb]{0.56,0.35,0.01}{#1}}
\newcommand{\ControlFlowTok}[1]{\textcolor[rgb]{0.13,0.29,0.53}{\textbf{#1}}}
\newcommand{\DataTypeTok}[1]{\textcolor[rgb]{0.13,0.29,0.53}{#1}}
\newcommand{\DecValTok}[1]{\textcolor[rgb]{0.00,0.00,0.81}{#1}}
\newcommand{\DocumentationTok}[1]{\textcolor[rgb]{0.56,0.35,0.01}{\textbf{\textit{#1}}}}
\newcommand{\ErrorTok}[1]{\textcolor[rgb]{0.64,0.00,0.00}{\textbf{#1}}}
\newcommand{\ExtensionTok}[1]{#1}
\newcommand{\FloatTok}[1]{\textcolor[rgb]{0.00,0.00,0.81}{#1}}
\newcommand{\FunctionTok}[1]{\textcolor[rgb]{0.13,0.29,0.53}{\textbf{#1}}}
\newcommand{\ImportTok}[1]{#1}
\newcommand{\InformationTok}[1]{\textcolor[rgb]{0.56,0.35,0.01}{\textbf{\textit{#1}}}}
\newcommand{\KeywordTok}[1]{\textcolor[rgb]{0.13,0.29,0.53}{\textbf{#1}}}
\newcommand{\NormalTok}[1]{#1}
\newcommand{\OperatorTok}[1]{\textcolor[rgb]{0.81,0.36,0.00}{\textbf{#1}}}
\newcommand{\OtherTok}[1]{\textcolor[rgb]{0.56,0.35,0.01}{#1}}
\newcommand{\PreprocessorTok}[1]{\textcolor[rgb]{0.56,0.35,0.01}{\textit{#1}}}
\newcommand{\RegionMarkerTok}[1]{#1}
\newcommand{\SpecialCharTok}[1]{\textcolor[rgb]{0.81,0.36,0.00}{\textbf{#1}}}
\newcommand{\SpecialStringTok}[1]{\textcolor[rgb]{0.31,0.60,0.02}{#1}}
\newcommand{\StringTok}[1]{\textcolor[rgb]{0.31,0.60,0.02}{#1}}
\newcommand{\VariableTok}[1]{\textcolor[rgb]{0.00,0.00,0.00}{#1}}
\newcommand{\VerbatimStringTok}[1]{\textcolor[rgb]{0.31,0.60,0.02}{#1}}
\newcommand{\WarningTok}[1]{\textcolor[rgb]{0.56,0.35,0.01}{\textbf{\textit{#1}}}}

\usepackage{graphicx}
% We will generate all images so they have a width \maxwidth. This means
% that they will get their normal width if they fit onto the page, but
% are scaled down if they would overflow the margins.
\makeatletter
\def\maxwidth{\ifdim\Gin@nat@width>\linewidth\linewidth
\else\Gin@nat@width\fi}
\makeatother
\let\Oldincludegraphics\includegraphics
\renewcommand{\includegraphics}[1]{\Oldincludegraphics[width=\maxwidth]{#1}}



\usepackage{setspace} 

\usepackage[export]{adjustbox}




\def\citeapos#1{\citeauthor{#1}'s (\citeyear{#1})}
\begin{document}  



\thispagestyle{plain} 
% Here's where it starts to differ from my statement template ----







\setlength{\tabcolsep}{1em}
\renewcommand{\arraystretch}{1.5}
\begin{tabular}{@{}ll@{}}
\textbf{FROM:} &  \\
% & \\
\textbf{TO:} &  \\
% & \\
\textbf{SUBJECT:} &  \\
% & \\
\textbf{DATE:} & 2025-11-23 \\
% & \\
\end{tabular}

\vspace{.5 em}


\hrule



\vspace{6 mm}
	


\section{Set Up \& Introduction}\label{set-up-introduction}

\subsection{The Data}\label{the-data}

\begin{itemize}
\item
  For this assignment, I will be analyzing data from the barroso2021
  data set. This data set is from the psymetadata package, and it
  observes the relationship between math achievement and math anxiety
  (\url{https://psycnet.apa.org/record/2020-80018-001}). Participants
  are in grades 1 through 6, and math ability is characterized as either
  low or not low.
\item
  The focus of this analysis will be analyzing differences in effect
  size across different parameters, where the absolute value was
  calculated, so values range from 0 (no effect) to 1 (strongest
  possible effect)
\end{itemize}

\subsection{Sample Size By Grade}\label{sample-size-by-grade}

Grade 5 has a significantly larger sample size than the other grades,
which will likely impact effect size measurements. Note: only grades 2
and 5 have students in the low math ability condition.

\subsection{Range of Effect Size
Values}\label{range-of-effect-size-values}

Effect sizes are organized on a scale from ``no effect'' to ``large
effect'':

\begin{itemize}
\item
  No effect: 0.0 until 0.2
\item
  Small effect: 0.2 until 0.5
\item
  Medium effect: 0.5 until 0.8
\item
  Large effect: 0.8 and above
\end{itemize}

\begin{Shaded}
\begin{Highlighting}[]
\NormalTok{barroso2021 }\SpecialCharTok{|\textgreater{}} 
  \FunctionTok{group\_by}\NormalTok{(yi) }\SpecialCharTok{|\textgreater{}}
  \FunctionTok{ggplot}\NormalTok{(}\FunctionTok{aes}\NormalTok{(}\AttributeTok{x=}\FunctionTok{abs}\NormalTok{(yi))) }\SpecialCharTok{+}
  \FunctionTok{geom\_bar}\NormalTok{(}\AttributeTok{fill=}\StringTok{"\#E5D6FF"}\NormalTok{) }\SpecialCharTok{+}
  \FunctionTok{labs}\NormalTok{(}
    \AttributeTok{x =} \StringTok{"Effect Size (Raw Value)"}\NormalTok{,}
    \AttributeTok{y =} \StringTok{"Count"}\NormalTok{,}
    \AttributeTok{title =} \StringTok{"Range of Effect Size Measurements (Absolute Value Scale)"}
\NormalTok{  ) }\SpecialCharTok{+}
  \FunctionTok{theme\_dark}\NormalTok{()}
\end{Highlighting}
\end{Shaded}

\pandocbounded{\includegraphics[keepaspectratio]{Assignment4_Stein_files/figure-latex/effect_size_groupings_raw-1.pdf}}

\subsection{Effect Size Distribution}\label{effect-size-distribution}

\begin{Shaded}
\begin{Highlighting}[]
\NormalTok{data\_by\_yi\_sorted }\SpecialCharTok{|\textgreater{}} 
  \FunctionTok{ggplot}\NormalTok{(}\FunctionTok{aes}\NormalTok{(}\AttributeTok{x=}\NormalTok{magnitude\_yi)) }\SpecialCharTok{+}
  \FunctionTok{geom\_bar}\NormalTok{(}\AttributeTok{fill=}\StringTok{"\#E5D6FF"}\NormalTok{) }\SpecialCharTok{+}
  \FunctionTok{labs}\NormalTok{(}
    \AttributeTok{x =} \StringTok{"Effect Size"}\NormalTok{,}
    \AttributeTok{y =} \StringTok{"Count"}\NormalTok{,}
    \AttributeTok{title =} \StringTok{"Range of Effect Sizes"}
\NormalTok{  ) }\SpecialCharTok{+}
  \FunctionTok{theme\_dark}\NormalTok{()}
\end{Highlighting}
\end{Shaded}

\pandocbounded{\includegraphics[keepaspectratio]{Assignment4_Stein_files/figure-latex/effect_size_distribution_visualization-1.pdf}}

\section{Visualizations \& Analysis}\label{visualizations-analysis}

\subsection{Sample Size - Grouped by Grade and Math
Ability}\label{sample-size---grouped-by-grade-and-math-ability}

\begin{Shaded}
\begin{Highlighting}[]
\CommentTok{\#create the sample size graph}
\NormalTok{sample\_size\_visualization }\OtherTok{\textless{}{-}}\NormalTok{ data\_by\_yi\_sorted }\SpecialCharTok{|\textgreater{}}
  \FunctionTok{group\_by}\NormalTok{(grade, low\_ability) }\SpecialCharTok{|\textgreater{}} 
  \FunctionTok{ggplot}\NormalTok{(}\AttributeTok{mapping=}\FunctionTok{aes}\NormalTok{(}\AttributeTok{x=}\NormalTok{grade, }\AttributeTok{fill=}\NormalTok{math\_ability)) }\SpecialCharTok{+}
  \FunctionTok{geom\_bar}\NormalTok{(}\AttributeTok{position=}\StringTok{"dodge"}\NormalTok{) }\SpecialCharTok{+}
  \FunctionTok{labs}\NormalTok{(}
    \AttributeTok{title =} \StringTok{"Distribution of Sample Sizes Per Grade and Math Ability Level"}\NormalTok{,}
    \AttributeTok{x =} \StringTok{"Count"}\NormalTok{,}
    \AttributeTok{y =} \StringTok{"Grade"}\NormalTok{,}
    \AttributeTok{fill =} \StringTok{"Math Ability"}
\NormalTok{  ) }\SpecialCharTok{+}
  \FunctionTok{theme\_dark}\NormalTok{()}
\end{Highlighting}
\end{Shaded}

\begin{Shaded}
\begin{Highlighting}[]
\CommentTok{\#Change colors and save the graph}
\NormalTok{visualization1 }\OtherTok{\textless{}{-}}\NormalTok{ sample\_size\_visualization }\SpecialCharTok{+} \FunctionTok{scale\_fill\_manual}\NormalTok{(}\AttributeTok{values =} \FunctionTok{c}\NormalTok{(}\StringTok{"Avg\_Plus\_Math\_Ability"} \OtherTok{=} \StringTok{"\#78FFEC"}\NormalTok{, }\StringTok{"Low\_Math\_Ability"} \OtherTok{=} \StringTok{"\#FF99EA"}\NormalTok{))}
\end{Highlighting}
\end{Shaded}

\pandocbounded{\includegraphics[keepaspectratio]{Assignment4_Stein_files/figure-latex/unnamed-chunk-1-1.pdf}}

\subsection{Effect Size Distributed By
Grade}\label{effect-size-distributed-by-grade}

\begin{Shaded}
\begin{Highlighting}[]
\CommentTok{\#Create and save the effect size per grade graph}
\NormalTok{visualization2 }\OtherTok{\textless{}{-}}\NormalTok{ data\_by\_yi\_sorted }\SpecialCharTok{|\textgreater{}} 
  \FunctionTok{group\_by}\NormalTok{(grade, magnitude\_yi) }\SpecialCharTok{|\textgreater{}} 
  \FunctionTok{summarize}\NormalTok{(}\AttributeTok{mean\_yi =} \FunctionTok{mean}\NormalTok{(abs\_yi), }\AttributeTok{count\_yi=}\FunctionTok{n}\NormalTok{()) }\SpecialCharTok{|\textgreater{}} 
  \FunctionTok{ggplot}\NormalTok{(}\FunctionTok{aes}\NormalTok{(}\AttributeTok{x=}\NormalTok{count\_yi, }\AttributeTok{y=}\NormalTok{magnitude\_yi))}\SpecialCharTok{+}
  \FunctionTok{geom\_col}\NormalTok{(}\AttributeTok{fill=}\StringTok{"\#E5D6FF"}\NormalTok{) }\SpecialCharTok{+}
  \FunctionTok{facet\_wrap}\NormalTok{(}\SpecialCharTok{\textasciitilde{}}\NormalTok{grade) }\SpecialCharTok{+}
  \FunctionTok{labs}\NormalTok{(}
    \AttributeTok{title =} \StringTok{"Effect Size Distributions"}\NormalTok{,}
    \AttributeTok{subtitle =} \StringTok{"For Each Grade (1 {-} 6)"}\NormalTok{,}
    \AttributeTok{x =} \StringTok{"Count"}\NormalTok{,}
    \AttributeTok{y =} \StringTok{"Effect Size"}
\NormalTok{  ) }\SpecialCharTok{+}
  \FunctionTok{theme\_dark}\NormalTok{()}
\end{Highlighting}
\end{Shaded}

\begin{verbatim}
## `summarise()` has grouped output by 'grade'. You can override using the
## `.groups` argument.
\end{verbatim}

\pandocbounded{\includegraphics[keepaspectratio]{Assignment4_Stein_files/figure-latex/unnamed-chunk-2-1.pdf}}

\subsection{Low vs Other Math Ability - Grades 2 and
5}\label{low-vs-other-math-ability---grades-2-and-5}

\begin{Shaded}
\begin{Highlighting}[]
\CommentTok{\#Create and save the graph for low vs high math level effect sizes in grades 2 and 5}
\NormalTok{effect\_size\_visualization }\OtherTok{\textless{}{-}}\NormalTok{ data\_by\_yi\_sorted }\SpecialCharTok{|\textgreater{}}
  \FunctionTok{filter}\NormalTok{(grade}\SpecialCharTok{==}\DecValTok{2} \SpecialCharTok{|}\NormalTok{ grade }\SpecialCharTok{==} \DecValTok{5}\NormalTok{) }\SpecialCharTok{|\textgreater{}} 
  \FunctionTok{group\_by}\NormalTok{(grade, magnitude\_yi, math\_ability) }\SpecialCharTok{|\textgreater{}} 
  \FunctionTok{summarize}\NormalTok{(}\AttributeTok{mean\_yi =} \FunctionTok{mean}\NormalTok{(abs\_yi), }\AttributeTok{count\_yi=}\FunctionTok{n}\NormalTok{()) }\SpecialCharTok{|\textgreater{}} 
  \FunctionTok{ggplot}\NormalTok{(}\FunctionTok{aes}\NormalTok{(}\AttributeTok{x=}\NormalTok{count\_yi, }\AttributeTok{y=}\NormalTok{magnitude\_yi, }\AttributeTok{fill=}\NormalTok{math\_ability))}\SpecialCharTok{+}
  \FunctionTok{geom\_col}\NormalTok{(}\AttributeTok{position=}\StringTok{"dodge"}\NormalTok{) }\SpecialCharTok{+}
  \FunctionTok{facet\_wrap}\NormalTok{(}\SpecialCharTok{\textasciitilde{}}\NormalTok{grade) }\SpecialCharTok{+}
  \FunctionTok{labs}\NormalTok{(}
    \AttributeTok{title =} \StringTok{"Effect Size Distributions"}\NormalTok{,}
    \AttributeTok{subtitle =} \StringTok{"For low vs average math abilities, grades 2 and 6"}\NormalTok{,}
    \AttributeTok{x =} \StringTok{"Count"}\NormalTok{,}
    \AttributeTok{y =} \StringTok{"Effect Size"}\NormalTok{,}
    \AttributeTok{fill =} \StringTok{"Math Ability"}
\NormalTok{  ) }\SpecialCharTok{+}
  \FunctionTok{theme\_dark}\NormalTok{()}
\end{Highlighting}
\end{Shaded}

\begin{verbatim}
## `summarise()` has grouped output by 'grade', 'magnitude_yi'. You can override
## using the `.groups` argument.
\end{verbatim}

\begin{Shaded}
\begin{Highlighting}[]
\CommentTok{\#Change colors and save the graph}
\NormalTok{visualization3 }\OtherTok{\textless{}{-}}\NormalTok{ effect\_size\_visualization }\SpecialCharTok{+} \FunctionTok{scale\_fill\_manual}\NormalTok{(}\AttributeTok{values =} \FunctionTok{c}\NormalTok{(}\StringTok{"Avg\_Plus\_Math\_Ability"} \OtherTok{=} \StringTok{"\#78FFEC"}\NormalTok{, }\StringTok{"Low\_Math\_Ability"} \OtherTok{=} \StringTok{"\#FF99EA"}\NormalTok{))}
\end{Highlighting}
\end{Shaded}

\pandocbounded{\includegraphics[keepaspectratio]{Assignment4_Stein_files/figure-latex/unnamed-chunk-3-1.pdf}}

\section{Conclusion}\label{conclusion}

There is a significantly smaller effect size for low math ability
participants compared to average-above average students. This is likely
impacted by the small sample sizes in these conditions. There is a less
obvious (but still important) difference in effect sizes for each grade.
Grade 5 has the highest number of scores for each effect size condition
except ``large effect''. This is likely due to its significantly higher
sample size than that of the other grades.




\end{document}

\makeatletter
\def\@maketitle{%
  \newpage
%  \null
%  \vskip 2em%
%  \begin{center}%
  \let \footnote \thanks
    {\fontsize{18}{20}\selectfont\raggedright  \setlength{\parindent}{0pt} \@title \par}%
}
%\fi
\makeatother
